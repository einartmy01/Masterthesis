\subsection{Crane Control}
Cranes are very different to railway, yet it has it similarities. As cranes are often also operated with joysticks and have certain set of possibilities in motion, it can be compared to trains in the aspect of how humans are effected mentally by remote operation.

\subsubsection{Brunnström 2020}

Brunnström et al. \cite{brunnström2020} study how latency affects the user experience when operating a forestry crane in a VR simulator. The goal is to understand how delays in the visual display and in the joystick controls influence task performance and comfort. The authors design three structured user studies where participants load logs in a VR environment using real crane joysticks and a head-mounted display. They systematically add different amounts of delay to the video update and to the joystick signals, and collect both performance metrics such as number of logs handled and subjective ratings of picture quality, responsiveness, comfort, immersion, and overall experience. Simulator Sickness Questionnaire (SSQ) scores are also gathered before and after each session to evaluate discomfort. The work provides a controlled methodology for evaluating latency sensitivity in immersive remote-operation tasks. 

